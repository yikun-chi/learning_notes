\chapter{PyTorch Tutorials} 

\section{Toy Example with PyTorch}
Here is the setup: 
    \begin{align*}
        & x = [x_1, x_2] \\
        & y = [y_1, y_2] \\
        & z = [x_1 y_1 , x_2 y_2 ]\\
        & s = x_1 y_1 + x_2 y_2 \\
    \end{align*}
We can find the derivative of $s$ w.r.t $x$: 
    \begin{align*}
        \partiald{x} s = [\partiald{x_1} s , \partiald{x_2} s ] = [y_1, y_2]
    \end{align*}

In PyTorch, this can be done as 
    \begin{python}
    x = torch.tensor([1,2], dtype = torch.float64)
    y = torch.tensor([3,4], dtype = torch.float64)
    x.requires_grad = True
    y.require_grad = True
    
    s = torch.sum(x * y)
    
    s.backward()
    
    x.grad
    \end{python}
    
\section{Reshaping}
We can use tensor.view to reshape the tensor. For example, if I have a tensor of length 16, I can reshape it to $2 \times 2 \times 4$. 
    \begin{python}
    x = torch.tensor([i for i in range(16)])
    print(x.shape)
    print(x.view(2,2,4))    
    \end{python}

\section{Transfer between Devices}
We can transfer a tensor from one device to the other. 
    \begin{python}
    device - torch.device("cuda")
    y = torch.ones(2).to(device)
    z = torch.ones(2).cuda()
    x = z.cpu()
    \end{python}