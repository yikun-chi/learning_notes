\chapter{Subgraph Mining} 

\section{Definition of Subgraphs} 

\subsection{Node-induced subgraph} 
Take subset of the nodes and all edges induced by the nodes: i.e.: $G'=(V', E')$ is a node induced subgraph iff
    \begin{align*}
        & V'\subseteq V\\
        & E' = \{ (u,v)\in E | u,v \in V' \}
    \end{align*}
This is also normally called node induced subgraph 

\subsection{Edge-induced subgraph} 
$G'=(V', E')$ is an edge induced subgraph iff
    \begin{align*}
        & E' \subseteq E \\
        & V' = \{ v\in V | (v,u) \in E' \textrm{for some $u$} \} 
    \end{align*}
So we determine edge first, and then add nodes that the edge belong to. 

\subsection{Graph Isomorphism} 
$G_1 = (V_1, E_1)$ and $G_2 = (V_2, E_2)$ are isomorphic if there exists a bijection $f:V_1 \to V_2$ such that $(u,v) \in E_1$ iff $(f(u), f(v))\in E_2$.

\subsection{Subgraph Isomorphism} 
$G_2$ is subgraph-isomorphic to $G_1$ ($G_1$ is a subgraph of $G_2$) if some subgraph of $G_2$ is isomorphic to $G_1$

\section{Network Motifs}
\subsection{Subgraph Frequency} 
\subsubsection{Graph Level Frequency} 
Let $G_Q$ be a small graph and $G_T$ be a target graph. The graph-level subgraph frequency is defined as number of unique subsets of nodes $V_T$ of $G_T$ for which the subgraph of $G_T$ induced by the nodes $V_T$ is isomorphic to $G_Q$.

\subsubsection{Node Level Frequency} 
Let $G_Q$ be a small graph, $v$ be a node in $G_Q$, and $G_T$ be a target graph. The node-level subgraph frequency is defined as: The number of nodes $u$ in $G_T$ for which some subgraph of $G_T$ is isomorphic to $G_Q$ and the isomorphism maps to node $u$ to $v$. $(G_Q, v)$ is called a node-anchored subgraph. 

\subsection{Random Graph Generation} 
\subsubsection{Erdos-Renyi(ER) random graphs}
$G_{n,p}$ is an undirected graph on $n$ nodes where each edge $(u,v)$ appears iid with probability $p$

\subsubsection{Configuration Model}
The goal is to generate a random graph with a given degree sequence $k_1, ...,k_N$.  

\paragraph{Method 1} 
    \begin{itemize}
        \item Generate nodes with spokes, and treat each node spoke as a mini-node
        \item Randomly pair up "mini" nodes. (Pick two from all the mini nodes that hasn't been paired, pair them)
        \item Connect node $A$ to node $B$ if there exists at least one mini-node connection. 
    \end{itemize}
The result will be a graph with approximately the same degree sequence as the given. Although in detail, we can have multiple mini-node connections between two nodes, or a inner-node mini-node connection, but the probability of those occurance gets really small for a large graph. 

\paragraph{Method 2: Switching } 
    \begin{itemize}
        \item Start from a given graph $G$
        \item Repeat the switching step $Q \cdot |E|$ times. Each time, select a pair of edges $A\to B$ and $C \to D$ at random, and exchange the endpoints to give $A \to D$ and $C \to B$. We exchange edges only if no multiple edges of self-edges are generated. 
    \end{itemize}

\subsection{Motif Significance} 
The overall process is 
    \begin{enumerate}
        \item Count motifs in the given graph $G_{real}$
        \item Generate a random graphs with similar statistics (number of nodes, edges, degree sequence and etc.), and count motifs in the random graphs 
        \item Use statistical measures to evaluate how significant is each motif (Z-score). Subgraph with high Z-score is the motif. 
    \end{enumerate}

\subsubsection{Z score and Network Significance Profile}
The $Z$ score for motif $i$ is defined as 
    \begin{align*}
        Z_i = \frac{N_i^{real} - \bar{N}_i^{rand}}{std(N_i^{rand}}
    \end{align*}
Network significance profile (SP) is defined as a vector of normalized z-scores: 
    \begin{align*}
        SP_i = \frac{Z_i}{\sqrt{\sum_j Z_j^2}}
    \end{align*}

\section{Neural Subgraph Representations and Counting} 

\subsection{Setup}
Given a large target graph (can be disconnected) and a query graph (connected), is query graph a subgraph in the target graph? We can use GNN to predict subgraph isomorphism by exploiting the geometric shape of embedding space to capture the properties of subgraph isomorphism. \\\par

Overall, we want to break the big graph into neighborhoods (containing node anchor), and compare the node-anchored embedding of the query graph and the neighborhoods. 

\subsection{Neighborhood Decomposition} 
\begin{itemize}
    \item For each node in $G_T$:
        \begin{itemize}
            \item Obtain a $k-hop$ neighborhood around the anchor. Can be performed using BFS, and $k$ is a hyper-parameter
        \end{itemize}
    \item For query $G_Q$, if the graph is large, do the same procedures. Otherwise, do not decompose
    \item Compute the embedding of the anchor nodes. 
\end{itemize}

\subsection{Order Embedding Space} 
We want to map graph $A$ to a point $z_A$ such that $z_A$ is non-negative in all dimensions. Also, if $B$ is a subgraph of $A$, then $z_B$ has lower value for all coordinates compare to $z_A$. 

\subsection{Loss function to satisfy order embedding space} 
We specify the order constraint: 
    \begin{align*}
        \forall_{i=1}^D \quad z_q[i] \leq z_u[i] \qquad iff  \quad G_Q \subseteq G_T
    \end{align*}
GNN Embeddings are learned by minimizing a max-margin loss
    \begin{align*}
        E(G_q, G_t) = \sum_{i=1}^D(max(0, z_q[i] - z_t[i]))^2
    \end{align*}
    
\subsection{Training Detail} 
Construct training examples $(G_q, G_t)$ where half the time, $G_q$ is a subgraph of $G_t$ and the other half it is not. Train on those examples by minimizing the max-margin loss:
    \begin{itemize}
        \item For positive examples: Minimize $E(G_q, G_t)$ 
        \item For negative examples: Minimize $\max(0, \alpha - E(G_q, G_t))$
    \end{itemize}


\subsection{Training Example Construction} 
We can get $G_T$ by choosing a random anchor $v$ and taking all nodes in $G$ within distance $K$ from $v$ to be in $G_T$ \\ \par

Positive examples: Sample induced subgraph $G_Q$ of $G_T$ with BFS sampling 
    \begin{itemize}
        \item Initialize $S = \{ v \} $, $V = \emptyset$
        \item Let $N(S)$ be all neighbors of nodes in $S$, at every step, sample 10\% of the nodes in $N(S) - V$, put them in $S$. Put the remaining nodes of $N(S)$ IN v
        \item After $K$ steps, take the sub-graph of $G$ induced by $S$ anchored at $q$
    \end{itemize}
Negative examples: "Corrupt" $G_Q$ by adding / removing nodes/edges so it is no longer a subgraph. 

\subsection{Prediction} 
Given query graph $G_q$ anchored at node $q$ and target graph $G_t$ anchord at node $t$. Output whether the query is a node-anchored subgraph of the target by comparing $E(G_q, G_t)$. If $E(G_q, G_t) < \epsilon$, then predict true. \\ \par

To check if $G_Q$ is isomorphic to a subgraph of $G_T$, repeat the checking for all $q\in G_Q, t\in G_T$
???


\section{Mining Frequent Subgraphs}
Goal: Find the frequency of the size $k$ motifs. \\
SPminer: Estimate frequency of $G_Q$ by counting the number of $G_{N_i}$ such that their embeddings $z_{N_i}$ staisfy $z_Q \leq z_{N_i}$
    \begin{itemize}
        \item Initial Step: Start by randomly picking a starting node $u$ in the target graph $G_T$, set $S = \{ u \}$
        \item Interatively: Grow a motif by iteratively choosing a neighbor in $G_T$ of a node in $S$ and add that node to $S$. We grow the motif $S$ until it reaches the target size $k$ 
    \end{itemize}