\chapter{Community Detection}

\section{Premier}
\subsection{Structural Definition of Edge Overlap} 
    \begin{align*}
        O_{ij} = \frac{\abs{(N(i) \cap N(j)) - \{ i, j \}}}{\abs{(N(i) \cup N(j)) - \{ i, j \}}}
    \end{align*}

\section{Louvain Algorithm} 
\subsection{Null Model: Configuration Model}
Given real $G$ on $n$ nodes and $m$ edges, construct rewired network $G'$ using the mini node method
    \begin{itemize}
        \item Same degree distribution but uniformaly random connections 
        \item Consider $G'$ as multigraph 
        \item The expected number of edges between nodes $i$ and $j$ of degree $k_i$ and $k_j$ equals $\frac{k_i k_j}{2m}$ where $m$ is the number of edges. ($2m$ number of mini nodes, $k_i$ chance of connection, each connection there is $\frac{k_j}{2m}$ chance of connecting to $j$)
    \end{itemize}
    
\subsection{Modularity} 
Modularity is a measure of how well a network is partitioned into communities. Given a partitioning of the network into groups disjoint $s\in S$: 
    \begin{align*}
        Q \propto \sum_{s\in S}[\textrm{(\# edges within group s})  - (\textrm{expected \# edges within group s})]
    \end{align*}
Note that this does not take into account edges between $s\in S$. The expected \# of edges is determined by null model. The modularity of partitioning $S$ of graph $G$ can be simplified as 
    \begin{align*}
        Q(G,S) = \frac{1}{2m}\sum_{s\in S} \sum_{i\in s} \sum_{j\in s}(A_{ij} - \frac{k_i k_j}{2m})
    \end{align*}
The range is $-1 \leq Q \leq 1$. This can also be extended to weighted edges. Generally, if modularity is greater than 0.3-0.7, it means significant community structure. 

\subsection{Louvain Algorithm Overview} 
Greedy algorithm for community detection with $O(n \log n)$ run time. It supports weighted graphs, and provides hierarchical communities. It initialize each node in a graph into a distinct community, then  iterates two phases 
    \begin{itemize}
        \item Phase 1: Modularity is optimized by allowing only local changes to node-communities memberships
        \item Phase 2: The identified communities are aggregated into super-nodes to build a new network. Edge weight between super-node is the weights sum of the edge weights in the original graph. 
    \end{itemize}

\paragraph{Phase 1} 
For each node $i$, the algorithm performs two calculations: 
    \begin{itemize}
        \item Compute the modularity delta $\nabla Q$ when putting node $i$ into the community of some neighbor $j$
        \item Move $i$ to a community of node $j$ that yields the largest gain in $\nabla Q$
    \end{itemize}
Repeat this process until no movement yields a gain. \\
$\nabla Q$ if node $i$ moves from community $D$ to $C$ can be decomposed into two terms: $\nabla Q(D \to i)$ and $\nabla Q(i \to C)$, i.e.: modularity change of removing $i$ from $D$ and change of adding $i$ to $C$. 
    \begin{align*}
        \nabla Q(i\to C) & \\ 
        & \sum_{in} \equiv \sum_{i,j \in C} A_{ij} \textrm{ Sum of link weights between nodes in C} \\
        & \sum_{tot} \equiv \sum_{i\in C} k_i \textrm{ Sum of all link weights of nodes in C} \\
        & Q(C) \\
        & = \frac{1}{2m}\sum_{i,j \in C}[A_{ij} - \frac{k_i k_j}{2m}] \\
        & = \frac{\sum_{i,j \in C} A_{ij}}{2m} - \frac{(\sum_{i\in C}k_i)(\sum_{j\in C} k_j)}{4m^2}\\
        & = \frac{\sum_{in}}{2m} - (\frac{\sum_{tot}}{2m})^2\\
        & k_{i, in} \equiv \sum_{j\in C} A_{ij} + \sum_{j\in C} A_{ji} \textrm{ Sum of link weights connecting node $i$ and $C$, double counted}\\
        & k_i \equiv \sum_{j} A_{ij} \textrm{ Sum of all link weights of node $i$}\\
        & Q_{before} = Q(C) + Q(\{ i \}) = \frac{\sum_{in}}{2m} - (\frac{\sum_{tot}}{2m})^2 + [0 - (\frac{k_i}{2m})^2]\\
        & Q_{after} = Q(C + \{ i \}) = \frac{\sum_{in} + k_{i, in}}{2m} - \left(\frac{\sum_{tot} + k_j}{2m} \right)^2\\
        & \nabla Q(i \to C) = Q_{after} - Q_{before}
    \end{align*}
$\nabla Q(D\to i)$ can be derived similarly. 


\section{Overlapping Community Detection: AGM}
Two step: 
    \begin{itemize}
        \item Define a generative model for graphs that is based on node community affiliations
        \item Given graph $G$, make the assumptionthat $G$ was generated by AGM, and find the best AGM that could have generated $G$
    \end{itemize}

\subsection{AGM Detail} 
Model parameters: node $V$, Communities $C$, Membership $M$. For each community $c$ has a single probability $p_c$ to indicate the chance of edge within community members. \\
The communities $C$ and nodes $V$ can form a bipartite graph with membership $M$ as edges. An edge exists indicates that node $u$ belong to community $c$. Nodes in community $c$ connect to each other by probability $p_c$. So $p(u,v) = 1 - \prod_{c\in M_u \cap M_v}(1-p_c)$

\subsection{Fitting AGM via MLE} 
Given a graph, we need to calculate $p(G|F)$ where $F$ is model parameters. 
    \begin{align*}
        P(G|F) = \prod_{(u,v)\in G} p(u,v) \prod_{(u,v) \notin G}(1 - p(u,v))
    \end{align*}

We want to to use $F = N \times C$ to represent membership association. $F_{ij}$ is a non-negative score indicating how likely node $i$ belong to community $j$. We can further model the edge probability as 
    \begin{align*}
        P_c(u,v) = 1 - \exp(-F_{uC} \cdot F_{vC})
    \end{align*}
In the case of multiple communicty, we have 
    \begin{align*}
        P(u,v) = 1 - \prod_{c\in \Gamma}(1-P_c(u,v)) = 1 - \exp(-F_u^Tf_v)
    \end{align*}
So we want to maximize the log likelihood using the given $p(u,v)$ expression. We have 
    \begin{align*}
        \sum_{(u,v)\in E} \log(1 - \exp(-F_u^TF_v)) - \sum_{(u,v)\notin E}F_u^TF_v
    \end{align*}

\section{Neural Overlapping Community Detection (NOCD)}
Key idea: Generate $F$ matrix using a GNN. e.g.: 
    \begin{align*}
        F= GCN(A,X) = \sigma(\tilde{A}\sigma(\tilde{A}W_1)W_2) & \tilde{A} = D^{-1}A
    \end{align*}
In order to deal with sparsity of real-world graphs, 
    \begin{align*}
        l(F) = \frac{1}{\abs{E}} \sum_{(u,v)\in E} \log(1 - \exp(-F_u^Tf_v)) - \frac{1}{n^2 - \abs{E}} \sum_{(u,v)\notin E}F_u^T F_v
    \end{align*}