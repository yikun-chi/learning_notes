\documentclass{article}
\usepackage[T1]{fontenc}
\usepackage[utf8]{inputenc}
\usepackage{mystyle}




\title{CS 224W Homework 3}
\author{Yikun Chi}

\begin{document}

\section*{Problem 1}
\subsection*{1.1}
    \begin{enumerate}
        \item Adding Node $A$
        \item Adding Node $B$, with edge decision $e_{B,A}=1$
        \item Adding Node $D$, with edge decision $e_{D,A} = 1$, $e_{D,B} = 0$
        \item Adding Node $C$, with edge decision $e_{C,B} = 1$, $e_{C,A} = 0$, $e_{C,D} = 0$
        \item Adding Node $E$, with edge decision $e_{E,B} = 1$, $e_{E,A} = 0$, $e_{E,D} = 1$,  $e_{E,C} = 0$
        \item Adding Node $F$, with edge decision $e_{F,A} = 0$, $e_{F,B} = 0$, $e_{F,C} = 1$,  $e_{F,D} = 0$, $e_{F,E} = 1$
    \end{enumerate}
\subsection*{1.2}
The two benefit is : 
    \begin{itemize}
        \item Only need to train for a specific ordering specified by BFS, instead of all possible combination of node permutations. 
        \item It reduces the number of edge predictions needed to make. Because at any point when adding a node, we only need to consider whether to add between that node and all the pre-existing nodes. 
    \end{itemize}

\newpage
\section*{Problem 2}
\subsection*{2.1}
Let $f_{A\to B}$ and $f_{B\to C}$ be the bijection mapping between nodes in A and B, and B and C. Now consider the bijection $f(v) = f_{B\to C}f_{A\to B}(v)$. This bijection maps every nodes in $V_A$ to subset of nodes in $V_C$. \\ \par

Now consider an arbitary edge $(u,v) \in E_A$, due to A being a subgraph of B, we know that edge $(f_{A\to B}(u), f_{A\to B}(v))$ exists in $B$ (definitin of graph-isomorphism). Due to $B$ being a subgraph of $C$, we know that edge $(f_{B \to C}(f_{A\to B}(u)), f_{B \to C}(f_{A\to B}(v)))$ exists in $C$\\ \par

In conclusion, there exists a bijection mapping between $V_A$ and a subset of nodes in $C$ such that the induced subgraph of $C$ is graph-isomorphic to $A$, hence $A$ is a subgraph of $C$. 


\subsection*{2.2}
Graph $A$ is a subgraph of graph $B$ implies that $|V_A| \leq |V_B|$, otherwise bijection wouldn't exists. Similiar, we have $|V_B| \leq |V_A|$. Combine together we have $|V_A| = |V_B|$. \\ \par

Given the same number of nodes, this means that the subgraph of $B$ induced by $\{ f(v) | v\in V_A \}$ is actually the entire graph. So this subgraph of $B$ is isomorphic to $A$ means that the graph $B$ is isomorphic to graph $A$. 


\subsection*{2.3}
Let $i$ be an arbitrary index in the embedding dimension. 
    \begin{align*}
        & \textrm{X is a common subgraph of A and B} \\
        & \Longrightarrow z_X[i] \leq z_A[i], z_X[i] \leq z_B[i]\\
        & \Longrightarrow z_X[i] \leq \min(z_A[i], z_B[i])
    \end{align*}
Because $i$ is chosen arbitrarily, we know this inequality is true for all indices in the embedding dimension, hence $z_x  \preccurlyeq \min(z_A, z_B)$


\subsection*{2.4}
Because A, B, C are not subgraphs of each other, and we have $z_A[0] > z_B[0] > z_C[0]$. This can be decomposed into $z_A[0] > z_B[0]$,  $z_A[0] > z_C[0]$, and  $z_B[0] > z_C[0]$. For the second dimension, each of this relations have to be reversed, because otherwise we would have a subgraph relation. So 
$z_A[1] \leq z_B[1] \leq z_C[1]$


\subsection*{2.5}
Given $z_A[0] > z_B[0] > z_C[0]$ and the fact that $A, B, C$ are not subgraph of each other, we also have $z_A[1] \leq z_B[1] \leq z_C[1]$. 
    \begin{align*}
        & \textrm{Let $X$ be the subgraph of $A$ and $C$.}\\
        & \Longrightarrow z_X \preccurlyeq \min(z_A, z_c)\\
        & \Longrightarrow z_X[0] \leq z_C[0], z_X[1] \leq z_A[1]
    \end{align*}
Let $Y, Z$ be the subgraph of $B$, but not subgraph of $A, C$. 
    \begin{align*}
        & \textrm{$Y$ be the subgraph of $B$} \\
        & \Longrightarrow z_Y[0] < z_B[0], z_Y[1] < z_B[1]\\
        & \Longrightarrow Z_Y[0] < Z_A[0], z_Y[1] < z_C[1] \\
        & \textrm{$Y$ is not the subgraph of $A$}\\
        & \Longrightarrow z_Y[1] \geq z_A[1] \\
        & \textrm{$Y$ is not the subgraph of $B$} \\
        & \Longrightarrow z_Y[0] \geq z_C[0]\\
        & z_X[0] \leq z_C[0], z_Y[0] \geq z_C[0]\\
        & \Longrightarrow z_X[0] \leq z_Y[0]\\
        & z_X[1] \leq z_A[1], z_Y[1] \geq z_A[1]\\
        & \Longrightarrow z_X[1] \leq z_Y[1]\\
        & \Longrightarrow z_X \preccurlyeq z_Y\\
    \end{align*}
We can follow the same logic to get $z_X \preccurlyeq z_Z $
\end{document}



