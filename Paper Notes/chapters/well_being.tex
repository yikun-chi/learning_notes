\chapter{Well Being}

\section{Digital Wellbeing as a Dynamic Construct} 
Author: Mariek M. P. Vaden Abeele, \href{https://academic.oup.com/ct/article/31/4/932/5927565}{Link} 

\paragraph{Intro Literature}
    \begin{itemize}
        \item Mobile connectivity
        \item Digital Wellbeing 
        \item Digital Wellbeing intervention inconclusive effect
        \item Mobile Connectivity paradox (the good and bad of mobile connectivity) 
    \end{itemize}

\paragraph{Consideration towards Defining Digital Wellbeing}
    \begin{itemize}
        \item No medicalization. Digital wellbeing is not lack of clinical addiction  
        \item Acnowledge both hedonic and eudemonic experience 
        \item Acknowledge temporal variability and person-specificity
        \item Acknoledgie ambivalence (digital connectivity have both good and bed effect)
    \end{itemize}

\paragraph{Definition of Digital Wellbeing}
Digital wellbeing is a subjective individual experience of optimal balance between the benefits and drawbacks to obtained from mobile connectivity. This experiential state is comprised of affective and cognitive appraisals of the integration of digital connectivity into ordinary life. People achieve digital wellbeing when experiencing maximal controlled pleasure and functional support, together with minimal loss of control and functional impairment. 

\paragraph{Factors that can influence Digital Wellbeing} A literature review on: 
    \begin{itemize}
        \item Person-specific
        \item Device-specific
        \item Context-specific
    \end{itemize}
    
\section{On Happiness and Human Potentials: A Review of Research on Hedonic and Eudaimonic Well-Being}
Author: Richard M. Ryan, Edward L. Deci
\href{https://pubmed.ncbi.nlm.nih.gov/11148302/}{Link}

\paragraph{Two Well-Being}
    \begin{itemize}
        \item Hedonic View: Well-being = happiness. Generally assessed through subjective well-being (SWB) 
        \item Eudaimonic View: Realization of one's true potential. Generally assessed through psychological well-being (PWB), which contains 6 aspect: autonomy, personal growth, self acceptance, life purpose, mastery, and positive relatedness. 
    \end{itemize}
Self Determination Theory embraced eudaimonic view, and defined more concrete psychology needs: autonomy, competence, and relatedness. 

\paragraph{Psychology of Well-Being}: A literature review on 
    \begin{itemize}
        \item personality, individual difference and well-being 
        \item emotions and well-being
        \item physical health and well-being
    \end{itemize}

\paragraph{Antecedents of Well-Being} A literature review on 
    \begin{itemize}
        \item social class and wealth
        \item attachment, relatedness 
        \item Goal pursuit
    \end{itemize}

\paragraph{Well-Being Across Time} A literature review on 
    \begin{itemize}
        \item Lifespan perspective
        \item cultural influences
    \end{itemize}