\chapter{Relationship}

\section{Liquidity and attachment in the mobile hookup culture. A comparative study of contrasted interactional patterns in the main uses of Grindr and Tinder}
Author: Christian Licoppe \href{https://www.tandfonline.com/doi/full/10.1080/17530350.2019.1607530}{Link}

\subsection{Abstract}
This study compares the interactional practices for the main types of uses of the mobile dating applications Grindr and Tinder. The analysis shows that in both cases, a majority of users share a similar orientation towards a linguistic ideology regarding ordinary conversation as a social institution, as topic-based, as allowing individuals to share and update knowledge so as to enable rapport and intimacy. However, Grindr and Tinder users take almost opposite conversational stances regarding the organization of casual hookups as sexual, one-off encounters with strangers. While many gay Grindr users have to chat to organize quick sexual connections, they become wary of the way their electronic conversations might waylay them into more personal relationships and they try to prevent this by developing an interactional genre made of laconic, fact-checking and very short exchanges. On the other hand, many heterosexual users on Tinder are looking to achieve topically-rich chat conversations. Their interactional dilemma, then, is the achievement of such topically-rich conversation, but with complete strangers. The interaction-oriented comparison provides a more detailed and subtle perspective of the alleged ‘liquefaction’ of romantic relationships into a casual hookup culture through the use of location-aware mobile dating applications

\subsection{Intro}
key concept 
    \begin{itemize}
        \item Location aware media / ecology 
        \item Purpose of Grindr, sexual script on Grindr 
        \item Purpose of Tinder / difference from Grindr 
        \item Study around online app relationship 
        \item Linguistic ideology of both apps 
    \end{itemize}

\subsection{Fieldwork } 
    \begin{itemize} 
        \item In-depth interview of twenty-three male users of Grindr in Paris 
        \item Interview and sample conversations (around 40)
    \end{itemize}
    
\subsection{Causal Hookups and First Conversations in Grindr} 
key concept 
    \begin{itemize}
        \item Grindr is for causal hookup 
        \item Conversation is short, and more in the mode of "hunter and prey", action driven, checklist like 
    \end{itemize}

\subsection{Chat conversations and encounters in Tinder}
key concept 
    \begin{itemize}
        \item heteronormative model of sexual relationships
        \item a ‘proper conversation’ is an expected fixture in Tinder post-match interactions
        \item balance between not too impersonal and not too personal, initiate and test a stream of relevant topics neither too personal or too intimate which might be taken up and lead to other subjects.
        \item More detailed analysis on the language, such as more elaborate reponse to yes no question. 
        \item Tinder conversationalists seem to maximize the topic and sequential affordances for the pursuit of the conversation.
    \end{itemize}

\subsection{Conclusion}
    \begin{itemize}
        \item Grindr and Tinder have different dominant interaction goal 
        \item Same linguistic ideology - interpersonal converstion. But different stances. Grindr users like separation between sexual gratification and personal relationships. Tinder is the opposite. 
    \end{itemize}
    
    
    
    
    
\section{Will You Go on a Date with Me? Predicting First Dates from Linguistic Traces in Online Dating Messages}
Author: Sabrina A. Huang and Jeffrey T. Hancock 
\href{https://journals.sagepub.com/doi/10.1177/0261927X211066612}{Link}

\subsection{Abstract}
From conveying intimacy (“I like you”) to irritation (“stop messaging me!”) and dissatisfaction (“I don’t think we’ll work out”), language use plays a fundamental yet often overlooked role in the initiation of relationships. In online dating, daters exchange messages to determine how interested they are in a partner and whether they would like to go on a first date with them. In two studies, we examined whether linguistic features present in online dating messages can predict whether a first date took place. In Study 1, we identified five interpersonal processes related to first date outcomes: investment, interdependence, emotional dynamics, decision-making, and coordination. In Study 2, we tested our hypotheses generated from Study 1 on a new dataset. Our results suggest that certain linguistic features within online dating messages can be used to predict above chance the likelihood of going on a first date.

\subsection{Intro}
\begin{itemize}
    \item Language and relationship are connected 
    \item Post-match conversation importance 
    \item Relevance of computer-mediated acquaintanceship 
\end{itemize}

\subsection{Language Use in Online Dating} 
\begin{itemize}
    \item Post-match conversation importance and some relevant research question
\end{itemize}

\subsection{Study 1} 
Key Concept: 
\begin{itemize}
    \item Five processes: investment, interdependence, emotional dynamics, decision-making, and coordination 
        \begin{itemize}
            \item Investment: longer messages may signal to the receiver that the sender is investing time and energy in pursuing the relationship
            \item Interdependence: relationship partners begin to form a joint identity and mentally represent the self as a self-and-partner pluralistic collective (operationalized by use of pronun) 
            \item Emotional Dynamics: Positive emotion 
            \item Decision-Making: messages with more language indicating decision-making processes, such as uncertainty, tentativeness, and negations, may indicate uncertainty about the partner and a greater desire for separation over integration
            \item Coordination: planning the date
        \end{itemize}
Methods: 
    \begin{itemize}
        \item $N = 348$, 172 males, 172 females, 2 something else, 2 prefer not to say 
        \item last 20 messages, 1 from a date, 1 from not a date 
    \end{itemize}
Analysis Plan: 
    \begin{itemize}
        \item T-test 
        \item FDR correction 
    \end{itemize}
\end{itemize}


\subsection{Study 2} 
Key Concept: 
\begin{itemize}
    \item Five Hypothesis: 
        \begin{itemize}
            \item Investment: Messages that take more effort to compose (e.g., more words, more words per sentence) will positively predict the likelihood of going on a first date.
            \item Interdependence: Greater use of language categories reflecting interdependence processes (e.g., first-person plural pronouns) will positively predict the likelihood of going on a first date.
            \item Enthusiasm: Greater use of language categories used to express enthusiasm (e.g., exclamation marks) will positively predict the likelihood of going on a first date.
            \item Negative Emotion: Greater use of language categories used to express negative emotions (e.g., negative emotion terms) will negatively predict the likelihood of going on a first date.
            \item Decision-Making: Greater use of language categories indicating uncertainty in the decision-making processes (e.g., negations, tentativeness, differentiation) will negatively predict the likelihood of going on a first date.
            \item Coordination: Greater use of language categories used to coordinate the logistics of setting up a first date (e.g., future focused words, space, time) will positively predict the likelihood of going on a first date.
        \end{itemize}
Methodology: 
    \begin{itemize}
        \item T-test, FDR Correction 
        \item Logistic regression 
    \end{itemize}
\end{itemize}
