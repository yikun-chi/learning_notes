\chapter{General Communication}

\section{Sixty years of quantitative communication research summarized lessons from 149 meta analyses}
Author: Stephen A. Rains, Timothy R. Levine \& Rene Weber
\href{https://www.tandfonline.com/doi/full/10.1080/23808985.2018.1446350}{Link}

\paragraph{Overall Summary} 
\begin{itemize}
    \item Six topic area considered: Persuasion, Media, Interpersonal, Instructional, Health, Organizational/group 
    \item Mean effect size $r=0.21$, and median $r=0.18$ 
    \item Table 3 contains a very comprehensive summaries. 
\end{itemize}




\section{The Ebb and Flow of Communication Research: Seven Decades of Publication Trends and Research Priorities}

Author: Nathan Walter, Michael J. Cody, Sandra J. Ball-Rokeach
\href{https://academic.oup.com/joc/article/68/2/424/4958955}{Link}

\paragraph{Overall Summary}
\begin{itemize}
    \item Analytic focus were divided into audience, message, source and policy. Audience and message remain as prominent categories. 
    \item See paper for detail summary in other areas such as analytic method, theoretical frameworks. 
    \item Table 3 has a popular theoretical framework tracking. 
\end{itemize}





\section{Problematization as a Methodology for Generating Research Questions}
Author: Mats Alvesson \& Jorgen Sandberg 

\paragraph{Iterative Process} 
\begin{enumerate}
    \item Identifying a domain of literature 
    \item Identifying and articulating the assumptions underlying this domain 
    \item Evaluating these assumptions 
    \item Developing an alternative assumption ground 
    \item Considering it in relation to its audience 
    \item Evaluating the alternative assumption ground 
\end{enumerate}
\paragraph{Types of Assumptions} 
\begin{enumerate}
    \item in-house : Assumptions exist within in a particular school of throughts. 
    \item root metaphor: Broader images of a particular subject matter
    \item paradigm: Ontological, epistemological, methodological assumptions. 
    \item ideology: Various political, moral, and gender related assumptions 
    \item field: High level broad assumptions
\end{enumerate}





\section{Communication in the communication sciences}
Author: Paisley, W 
Cite: In B. Dervin and M.Voight (eds.), Progress in the Communication Sciences
\subsection{Key Concept}
Level fields (Anthropology, Sociology, Pscyhology, Physiology, ...) and variables fields together form a grid. 
    \begin{itemize}
        \item Level fields: Physical Science -> Biological Science -> Behavioral Science
        \item Variable fields: Fundamental fields: cybernetics, systems research, communication research. Higher-order variable fields: Political science, economics, business, education etc. 
    \end{itemize}

Part 2 includes a more detailed breakdown of communication journals and institutions. 





\section{Disciplines, Intersections, and the Future of Communication Research}
Author: Susan Herbst 
\href{https://academic.oup.com/joc/article/58/4/603/4098348}{Link}

\subsection{Key Concept: Comm as postdiscipline}
Communication as postdisciplinary, which is defined as "retains nothing of the notion of a shared consciousness, or of a shared objective that brings together a broad range of discrete studies. Instead it suggests that the organizing structures of disciplines themselves will not hold. ONly conditional conjunctions of social and intellectual force exist." In another word, it is more topic driven then discipline driven. \\

The danger of postdiscipline includes: reinventing the wheel (ignore old fields during literature review) and quality of research due to peer review expertise. \\

What to do next? Publish in noncomm journal, align with more influential publications / to get them report comm study, and understand more about peer review process in comm. 





\section{The "Concept" of Communication} 
Author: Frank E. X. Dance 
\href{https://academic.oup.com/joc/article/58/4/603/4098348}{Link}
\subsection{15 Conceptual Component of Communication}
    \begin{itemize}
        \item Symbols/ Verbal / Speech: Communication is the verbal interchange of thought or idea
        \item Understanding: Communication is the process by which we understand others and in turn endeavor to be understood by them.
        \item Interaction / Relationship/ Social Process: Interaction, even on the biological level, is a kind of communication 
        \item Reduction of uncertainty: Communication arises out of the need to reduce uncertainty to act effectively, to defend or strengthen the ego. 
        \item Process: the transmission of information, ideas, emotions skills, etc., by the use of symbols - words, pictures, figures, graph etc. It is the act or process of transmission that is usually called communication 
        \item Transfer/ Transmission / Interchange: the connecting thread appears to be the idea of something’s being transferred from one thing, or person, to another. We use the word “communication” sometimes to refer to what is so transferred, sometimes to the means by which it is transferred, sometimes to the whole process. In many cases, what is transferred in this way continues to be shared; if I convey information to another person, it docs not leave my own possession through coming into his. Accordingly, the word “communication” acquires also the sense of participation
        \item Linking / Binding: Communication is the process that links discontinuous parts of the living world to one another
        \item Commonality: It (communication) is a process that makes common to two or several what was the monopoly of one or some
        \item Channel / Carrier / Means / Route: the means of sending military messages, orders, etc. as by telephone, telegraph, radio, couriers
        \item Replicating Memories: Communication is the process of conducting the attention of another person for the purpose of replicating memories
        \item Discriminative Response / Behavior Modifying / Response / Change: ... is the discriminatory response of an organism to a stimulus 
        \item Stimuli: Every communication act is viewed as a transmission of information, consisting of a discriminative stimuli, from a source to a recipient
        \item Intentional: In the main, communication has as its central interest those behavioral situations in which a source transmits a message to a rcceiver(s) with conscious intent to affect the latter’s behaviors.
        \item Time / Situation: The communication process is one of transition from one structured situation-as-a-whole to another, in preferred design.
        \item Power: Communication is the mechanism by which power is exerted. 
    \end{itemize}
Three points of conceptual cleavage to consider: 
    \begin{itemize}
        \item Level of observation 
        \item The presence or a absence of intent on the part of the sender 
        \item The normative judgment of the act 
    \end{itemize}
    



\section{Institutional Sources of Intellectual Poverty in Communication Research}
Author: John Durham Peters
Link: \href{https://journals.sagepub.com/doi/10.1177/009365086013004002}{Link}
\subsection{Summary}
Key Idea: Communication as a field failed to define its "mission, subject matter, and relation to society in a coherent way". It is now more administratively instead of conceptually defined as a discipline. 

\begin{itemize}
    \item Professional Social Science and America Democracy: social science in U.S. is deep rooted in the promise of bring democracy. 
    \item The Rise of Big Social Science: After 1930s, the prophetic idea of social science and democracy disappears. People start working on topics of immediate social concern. 
    \item Source of intellectual poverty 1, institutionalization: so much of the original development are driven by policy science pre and post WW2. 
    \item The Berelson-Schramm debate: Transformation of communication research from an intellectual to an institutional entity. 
    \item Source of intellectual poverty 2, the use of information theory: Information theory was popular, but not used in depth. It never really helped with the development of Comm. 
    \item Source of intellectual poverty 3, self-reflection as institutional apologetics: Most of the self reflection are used to justify institutional existance instead of comming up with a cohesive theory 
    \item Communication as a state: later listed four reasons why this is bad. 
    \item The problem of professional status: uncertainty about what comm scholars are. 
\end{itemize}




\section{Communication as an Academic Discipline: A Dialogue} 
Author: Everett M. Rogers and Steven H. Chaffee 
Link: \href{https://academic.oup.com/joc/article-abstract/33/3/18/4282731?redirectedFrom=fulltext}{Link}

\subsection{Summary}
    \begin{itemize}
        \item The history of communication research has largely been one of response to technological innovations, with the study of each new mass medium going through the same series of phases. 
        \item Communication went through a changing landscape w.r.t its relation with other social sciences 
        \item there is a big division of interpersonal vs. mass media channels. 
    \end{itemize}
    




\section{Historical Trends in Research on Children and the Media:1900-1960}
Author: Ellen Wartella and Byron Reeves 
Link: \href{https://academic.oup.com/joc/article-abstract/35/2/118/4282851?redirectedFrom=fulltext}{Link}

\subsection{Summary}
Using children in media as an example, it shows that "The history of communication research has largely been one of response to technological innovations, with the study of each new mass medium going through the same series of phases. "





\section{Communication Research: A History} 
Author: Jesse G. Delia 

\subsection{Summary} 
Comm history is divided into three periods: 1900 to 1940s, 1940 to 1965, and onwards. Four major influences are 
    \begin{itemize}
        \item Fragmentation of concern with communication across diverse disciplines and fields of interest  
        \item Study of the media of mass communication 
        \item Role of public communication media in social and political life 
        \item Evolution of professional practices within and across the social science disciplines
    \end{itemize}

\subsection{Period 1: 1900 to 1940} 
Communication grow in 19th to early 20th century as the society grows. Some of the key areas include 
    \begin{itemize}
        \item Political Issues 
            \begin{itemize}
                \item Political and social themes in public communication 
                \item Content analysis and the quantitative analysis of messages 
                \item Public opinion research 
                \item The broader landscape of political communication. 
            \end{itemize}
        \item Social Life 
            \begin{itemize}
                \item Chicago school of sociooogy 
                \item The broader practice of "Chicago-style" research 
                \item Social Psychological Analyses 
            \end{itemize}
        \item Education
        \item Commerical 
    \end{itemize}

\subsection{Period 2: Consolidation in 1940 to 1965}
    \begin{itemize}
        \item WW2
        \item Field's consolidation 
    \end{itemize}
    
\subsection{Relation and Merging with Journalism School}





\section{Combining, Distinguishing, and Generating Theories in Communication: A Domains of Analysis Framework }
Author: Clifford Nass, Byron Reeves \href{https://journals.sagepub.com/doi/abs/10.1177/009365091018002006}{Link}

\section{Traditional Approach}
Traditional comm scholars tend to assume: 
    \begin{itemize}
        \item Traditional levels are a necessary distinction for theorizing and operationlizing 
        \item Theories and operations can and should be mixed levels. 
    \end{itemize}


\section{Theoretical Domain and Operational Domain} 
Theoretical domain of analysis is defined as :"class of entities that, for each and every characteristic, is conceptualized to have one and only one of the values associated with that characteristic". Operational domain is the same definition, but with added "take on one and only one value as determined by the measurement procedure."  "A theory is composed of variables, constants, and relationships between variables". 

\paragraph{Four Benefit of Domain Analysis} 
\begin{itemize}
    \item Domains define the set of entities to which theories and operations can refer 
    \item Domains are consistent with traditional level (bio, psy, socio) but permit partial inclusion and exclusion of entities from multiple levels 
    \item Domains can differentiate and link theories and operations
    \item Domains explain when theories can and cannot "cross" levels, and specify how to adjust inappropriate "cross-level" theories. 
\end{itemize}

\paragraph{Core Restriction}
\begin{itemize}
    \item Any theory that specifies one variable at one domain of analysis and another variable at another domain of analysis is inherently ambiguous. 
\end{itemize}