\chapter{Statistics}

\section{Frequentist Parametric Estimators}

\subsection{Maximum Likelihood} 
Estimate the parameter $\hat{\theta}$ via the value that maximize the likelihood of observing the dataset. Generally we ended up maximizing the log likelihood. 

\paragraph{Example: MLE for Uniform Distribution} 
Assume the data are drawn from uniform distribution in $[0, \theta]$
    \begin{align*}
        L_n(\theta) 
        & = \prod_{i=1}^n \frac{\indic{0 \leq x_i \leq \theta}}{\theta}\\
        & = \indic{\theta \geq \max{X_i}} * \frac{1}{\theta^n}
    \end{align*}
So the $\theta$ that minimze the likelihood is the minimum $\theta$ which satisfies the indicator variable. Hence $\hat{\theta} = \max{X_i}$. Notice that this particular estimator actually converges in $n$ to exponential distribution instead of the slower $\sqrt{n}$ with CLT. 

\subsection{Method of Moments}
The general method of moments is pick $k(x)$ such that $E_{\theta_1}[k(x)] = E_{\theta_2}[k(x)] \Longleftrightarrow \theta_1 = \theta_2$. Then we have 
    \begin{align*}
        \frac{1}{n} \sum k(x_i) \approx E_\theta[k(x)]
    \end{align*}

So we use the theoretical result of the first $t$ moments and equalizing them to the empirical sample moments in order to solve for $\theta$. 

\subsection{Estimating Equations} 
Estimating equation generalizes both Method of Moments and MLE. 
    \begin{align*}
        E_{\theta_1}[g(\theta_2, X)] = 0 \text{ iff } \theta_1 = \theta 2
    \end{align*}
So then we can estimate $\theta_1$ with the root $\hat{\theta}$ of the equation: 
    \begin{align*}
        \frac{1}{n} \sum_{i=1}^n g(\hat{\theta}, X_i) = 0
    \end{align*}
In MLE, we have 
    \begin{align*}
        g(\theta, x) = \frac{\nabla_\theta f(\theta,x)}{f(\theta,x}
    \end{align*}
where as in MoM, we have 
    \begin{align*}
        g(\theta, x) = E_\theta[k(X)] - k(x)
    \end{align*}
\subsubsection{Large Sample Behavior of Estimator} 
In estimating equation framework, assuming consistency, i.e.: $\hat{\theta}_n \overset{P}{\to} \theta^*$ the estimator is a root of:
    \begin{align*}
        \frac{1}{n} \sum g(\hat{\theta}_n, x_i) = 0
    \end{align*}
So now we can expand to look at the convergence 
    \begin{align*}
        & \frac{1}{n} \sum \left( g(\hat{\theta}_n, x_i) - g(\theta^*, x_i)\right) = -\frac{1}{n} \sum g(\theta^*, x_i)\\
        & \frac{1}{n}\sum g'(\xi_n, x_i)(\hat{\theta}_n - \theta^*) = -\frac{1}{n} \sum g(\theta^*, x_i) \tag{By mean value theorem} \\
        & \frac{1}{n}\sum g'(\xi_n, x_i)\sqrt{n}(\hat{\theta}_n - \theta^*) = -\frac{1}{\sqrt{n}} \sum g(\theta^*, x_i)\\
        & \frac{1}{n}\sum g'(\xi_n, x_i)\sqrt{n}(\hat{\theta}_n - \theta^*) = -\frac{1}{\sqrt{n}} \left(\sum g(\theta^*, x_i) - n * E[g(\theta^*, x_i)] \right) \tag{Expectation = 0 due to estimating equation setup} \\
        & \frac{1}{n}\sum g'(\xi_n, x_i)\sqrt{n}(\hat{\theta}_n - \theta^*) = N(0, Var_{\theta^*}(g(\theta^*, x_i)) \tag{By CLT}\\
        & \frac{1}{n}\sum g'(\xi_n, x_i)\sqrt{n}(\hat{\theta}_n - \theta^*) = N(0, E_{\theta^*}[g(\theta^*, X_i)^2]
    \end{align*}
Now couple observations: 
    \begin{enumerate}
        \item $\frac{1}{n}\sum g'(\xi_n, x_i) \overset{P}{\to} E_{\theta^*}[g'(\theta^*, X_1)]$ (shown later)
        \item $\hat{\theta}_n - \theta^* \overset{P}{\to} 0$ by consistency
        \item The RHS converged to a normal random variable 
    \end{enumerate}
So we can use Slutsky's algorithm to actually establish the convergence 
    \begin{align*}
        \sqrt{n}(\hat{\theta}_n - \theta^*) 
        & \Rightarrow \frac{N(0, E_{\theta^*}[g(\theta^*, X_i)^2] }{E_{\theta^*}[g'(\theta^*, X_i)]}\\
        & \overset{D}{=} N\left(0, \frac{E_{\theta^*}[g(\theta^*, X_i)^2]}{E_{\theta^*}[g'(\theta^*, X_i)]^2} \right)
    \end{align*}
Therefore, we can establish the confidence interval for estimating equation estimator $[\hat{\theta}_n - 2 \frac{\hat{S}_n}{\sqrt{n}}, \hat{\theta}_n + 2 \frac{\hat{S}_n}{\sqrt{n}}]$ where 
    \begin{align*}
        \hat{S}_n 
        & = \sqrt{\frac{E_{\theta^*}[g(\theta^*, X_i)^2]}{E_{\theta^*}[g'(\theta^*, X_i)]^2}} \\
        & \approx \sqrt{\frac{\frac{1}{n} \sum g(\hat{\theta}_n, x_i)^2}{(\frac{1}{n}\sum g'(\hat{\theta}_n, x_i))^2}}
    \end{align*}

Now just the last bits of clean up: Proving $\frac{1}{n}\sum g'(\xi_n, x_i) \overset{P}{\to} E_{\theta^*}[g'(\theta^*, X_1)]$. Notice that $g'(\xi_n, x_i)$is not iid becuase $\xi_n$ is constructed based on $\hat{\theta}_n$, which is dependent on all $x_i$. We can do something similar by subtracting certain terms from both side. i.e.: We want to show  
    \begin{align*}
        & \frac{1}{n}\sum g'(\xi_n, x_i) - \frac{1}{n} \sum g'(\theta^*, x_i)\overset{P}{\to} E_{\theta^*}[g'(\theta^*, X_1)] -  \frac{1}{n} \sum g'(\theta^*, x_i) \\
        & \frac{1}{n}\sum g''(\beta_n, x_i)(\xi_n - \theta^*) \overset{P}{\to} E_{\theta^*}[g'(\theta^*, X_1)] -  \frac{1}{n} \sum g'(\theta^*, x_i) \tag{By mean value theorem}
    \end{align*}
Notice that the right hand side converges to 0 by SLLN. So now we just need to show that the left hand side also converges to 0. Notice that $\beta_n$ is between $\hat{\theta}_n$ and $\xi_n$. If we assume that the second derivative is global bounded, hence $\sup_\theta \abs{g''(\theta,x)} \leq s(x)$, then we have 
    \begin{align*}
        \abs{\frac{1}{n}\sum g''(\beta_n, x_i)(\xi_n - \theta^*)} \leq \frac{1}{n}\sum s(x_i) \abs{\xi_n - \theta^*}
    \end{align*}
Notice on the right hand side, the first part converges to $E_{\hat{\theta}_n}[s(X_i)]$ and the second part converges to 0. So by Slutsky the product converge to 0. 


\subsubsection{Asymptotic Variance of MLE in Estimating Equation Format} 
In MLE formulation of Estimating equation, we already have 
    \begin{align*}
        & E_{\theta^*}[\frac{f'(\hat{\theta}_n, x_1)}{f(\hat{\theta}_n, x_1)}] = 0 \\
        & g(\theta, x) = \frac{f'(\theta, x_1)}{f(\theta, x_1)}\\
        & g'(\theta, x) = \frac{f''(\theta,x)}{f(\theta,x)} - (\frac{f'(\theta,x)}{f(\theta,x})^2 \\
        & E_{\theta^*}[g(\theta^*, X_1)^2] = E[\left(\frac{f'(\theta, x_1)}{f(\theta, x_1)}\right)^2]\\
    \end{align*}
First we simplify $E[g'(\theta,x)]$: 
    \begin{align*}
        E_{\theta^*}[g'(\theta,x)]
        & = E_{\theta^*}[\frac{f''(\theta,x)}{f(\theta,x)} - (\frac{f'(\theta,x)}{f(\theta,x)})^2] \\
        & = E_{\theta^*}[\frac{f''(\theta,x)}{f(\theta,x)}] - E[ (\frac{f'(\theta,x)}{f(\theta,x)})^2]\\
        & = \int_{\mathbbm{R}} \frac{f''(\theta,x)}{f(\theta,x)} f(\theta, x) \,dx - E[ (\frac{f'(\theta,x)}{f(\theta,x)})^2]\\ 
        & = \int_{\mathbbm{R}} \partiald{\theta} \partiald{\theta} f(\theta, x) \,dx - E[ (\frac{f'(\theta,x)}{f(\theta,x)})^2]\\ 
        & = \partiald{\theta} \partiald{\theta} \int_{\mathbbm{R}} f(\theta, x) \,dx - E[ (\frac{f'(\theta,x)}{f(\theta,x)})^2]\\ 
        & = \partiald{\theta} \partiald{\theta} 1 - E[ (\frac{f'(\theta,x)}{f(\theta,x)})^2] \\
        & = - E[ (\frac{f'(\theta,x)}{f(\theta,x)})^2]
    \end{align*}
So now we can start to simplify the previous variance $\frac{E_{\theta^*}[g(\theta^*, X_i)^2]}{E_{\theta^*}[g'(\theta^*, X_i)]^2}$: 
    \begin{align*}
        \frac{E_{\theta^*}[g(\theta^*, X_i)^2]}{E_{\theta^*}[g'(\theta^*, X_i)]^2}
        & = E[\left(\frac{f'(\theta, x_1)}{f(\theta, x_1)}\right)^2] * \frac{1}{E[ (\frac{f'(\theta,x)}{f(\theta,x)})^2]^2}\\
        & = \frac{1}{E[ (\frac{f'(\theta,x)}{f(\theta,x)})^2]}\\
    \end{align*}
Hence the asymptotic variance $\hat{\theta}_n$ for MLE is 
    \begin{align*}
        \frac{1}{n} \frac{1}{E_{\theta^*}\left[ \left(\frac{f'(\theta,x)}{f(\theta,x)}\right)^2 \right]}
    \end{align*}
    
    
\subsection{Cramer-Rao Bound}
Cramer-Rao bound establishes that MLE is the best estimator for unbiased estimator. Let $\hat{\theta}_n = w_n(X_1, ..., X_n)$ and assume that the estimator is unbiased at every point $\theta \in \Lambda$. We have 
    \begin{align*}
        & \theta = E_\theta[w_n(X_1, ..., X_n)]
        = \int_{\mathbbm{R}^n} w_n(x_1, ..., x_n) \prod_{i=1}^n f(\theta, x_i) \, dx_1, ...dx_n \\
        & \Longrightarrow  \int_{\mathbbm{R}^n} w_n(x_1, ..., x_n) \sum_{i=1}^nf'(\theta,x_i)\prod_{j\neq i}f(\theta,x_j)\, dx_1, ...dx_n = 1 \tag{Chain rule and derivative both side} \\
        & \Longrightarrow \int_{\mathbbm{R}^n} w_n(x_1, ..., x_n) \sum_{i=1}^n \frac{f'(\theta,x_i)}{f(\theta,x_i)} * \prod_{j=1}^n f(\theta,x_j) \, dx_1, ...dx_n = 1\\
        & \Longrightarrow E_\theta\left[w_n(X_1, ..., X_n) \sum_{i=1}^n \frac{f'(\theta,X_i)}{f(\theta,X_i)} \right] = 1\\
        & A \equiv w_n(X_1, ..., X_n)  \\
        & B \equiv  \sum_{i=1}^n \frac{f'(\theta,X_i)}{f(\theta,X_i)}
    \end{align*}
Notice that we care about the variance of our estimator, hence $E[A^2] = E_\theta[w_n(X_1, ..., X_n)^2]$. By Cauchy Schwartz we have 
    \begin{align*}
        \sqrt{E[A^2] E[B^2]} \geq \abs{E[AB]} = 1
    \end{align*}
Notice that 
    \begin{align*}
        E_[B^2] 
        & = E\left[ \left( \sum_{i=1}^n \frac{f'(\theta,X_i)}{f(\theta,X_i)} \right)^2 \right]\\
        & = Var\left(\sum_{i=1}^n \frac{f'(\theta,X_i)}{f(\theta,X_i)}  \right) \tag{Mean 0} \\
        & = n * Var\left(\frac{f'(\theta,X_1)}{f(\theta,X_1)} \right) \\
        & = n * E\left[\left(\frac{f'(\theta,X_1)}{f(\theta,X_1)} \right)^2 \right]
    \end{align*}
So we have 
    \begin{align*}
        E_\theta[\hat{\theta}^2] = E[A^2] \geq \frac{1}{n * E\left[\left(\frac{f'(\theta,X_1)}{f(\theta,X_1)} \right)^2 \right]}
    \end{align*}
We can look at the connection to fisher information. Individual variable in B can be seen as the score function 
    \begin{align*}
        E[\partial{\theta} \log f(X, \theta)]
        & = \int_{\mathbbm{R}} \frac{f'(x, \theta}{f(x,\theta)} f(x,\theta) \, dx\\
        & =  \int_{\mathbbm{R}} \partiald{\theta} f(x,\theta) \, dx \\
        & = 0
    \end{align*}
and the variance is the of the score, which is $E\left[\left(\frac{f'(\theta,X_1)}{f(\theta,X_1)} \right)^2 \right]$. 



\section{Frequentist Non-parametric Perspective} 

\subsection{Setup and Glivenko–Cantelli theorem}
In frequentist non-parametric perspective, we have $X_i$ iid from an unknown distribution $F$. We attach mass $\frac{1}{n}$ to each point $X_i$. So the empirical CDF is defined as 
    \begin{align*}
        F_n(x) = \frac{1}{n} \sum \indic{x_i \leq x}
    \end{align*}
From SLLN, we know that as $n \to \infty$, $\forall x\in \mathbbm{R}$ (note that below theorem is for a given $x$), we have
    \begin{align*}
        F_n(x) \overset{a.s.}{\to} F(x) 
    \end{align*}
But Glivenko-Cantelli Theorem provides a stronger point that says 
    \begin{align*}
        \sup_{x \in \mathbbm{R}}\abs{F_n(x) - F(x)} \overset{a.s.}{\to} 0
    \end{align*}
The proof can be found \href{chrome-extension://efaidnbmnnnibpcajpcglclefindmkaj/https://home.uchicago.edu/~amshaikh/webfiles/glivenko-cantelli.pdf}{here}. \\

Another restatement of GC theorem is that 
    \begin{align*}
        & \sup_{h \in S} \abs{\int h(x) F_n(dx) - \int h(x) F(dx)} \overset{a.s.}{\to} 0 \\
        & S = \{ h: h(y) = \indic{y \leq x} \text{ for some } x \in \mathbbm{R} \}
    \end{align*}
Notice that this result also applies to a broader function space (GC class), but not to every function. 

\subsection{Plug-in Principle}
For any fixed function $h$, we have that 
    \begin{align*}
        \int h(x) F_n(dx) = \frac{1}{n} \sum h(x_i) \to \int h(x)F(dx)
    \end{align*}
So if we want to estimate a function of the distribution $T(F)$, we can just use the empirical distribution $T(F_n)$. So if we want to calculate the median of $F$, we can esimate it with sample median. 
    \begin{align*}
        & T(F) = \int_{\mathbbm{R}} h(x) F(dx) = E[h(x)] \\
        & T(F_n) = \frac{1}{n} \sum h(x_i) \\
        & \sqrt{n} (T(F_n) - T(F)) \Rightarrow \sqrt{Var(h(x))}N(0,1) \tag{CLT} \\
        & \Rightarrow CI: [T(F_n) \pm Z \frac{\sigma(F_n)}{\sqrt{n}}]
    \end{align*}


\subsection{Skorokhod space}
A Skorokhod space e.g.: $D(-\infty, \infty)$ is a space of all functions $Z$ such that 
    \begin{itemize}
        \item $Z$ is right continuous: $\lim_{t \downarrow s}z(t) = z(s)$
        \item $Z$ has left limit wverywhere: $\lim_{t \uparrow s}z(t) = z(s-)$ exists
    \end{itemize}
$Z$ is defined on the interval of the real line (e.g.: $(-\infty, \infty)$), and takes values on the real line or some metric space. 

\subsection{Alternative Definition of Weak Convergence} 
An alternative definition of weak convergence is $Y_n \Rightarrow Y_\infty$ in $\mathbbm{R}$ iff: 
    \begin{align*}
    E[g(Y_n)] \to E[g(Y_\infty]
    \end{align*}
for all $g$ that are bounded converges. This allows us to think about convergence of random variable when they are in a more complicated space such as the Skorokhod space. 

\subsection{Convergence in abstract metric space}
Given a metric space $S$ and a distance measure $d: S \times S \to \mathbbm{R}$. The distance $d$ has to satisfies: 
    \begin{itemize}
        \item $d(x,x) = 0$
        \item $d(x,y) = d(y,x)$
        \item $d(x,y) \leq d(x,z) + d(z,y)$
    \end{itemize}
then we have 
    \begin{align*}
        y_n \to y_\infty \text{ iff } d(y_n, y_\infty) \to 0 \text{ as } n \to \infty
    \end{align*}
    
\subsection{Completeness}
Cauchy Sequence is defined as :
    \begin{align*}
        \forall \epsilon > 0 \,, \exists n = n(\epsilon) \, s.t. \, m_1, m_2 \geq n(\epsilon) \Longrightarrow \abs{X_{m_1} -X_{m_2}} < \epsilon
    \end{align*}
The completeness of real line means that Cauchy sequence on real line always converges to an element in real. Equivalently convergence sequence in real line has to be a Cauchy Sequence. This definition of completeness can be extended to abstract metric space $S$ by replacing the definition of $-$ in Cauchy sequence with general distance measurement. 

\subsection{Separability} 
Metric space $S$ is separable iff a countable set of element in $S$ can be used to approximate all element in $S$ (have a dense countable subset). 



\subsection{Convergence in Polish Space} 
Let metric space $S$ with measurement $d$ be a complete separable metric space (Polish Space), then 
    \begin{align*}
        Y_n \Rightarrow Y_\infty \text{ in } (s,d)
    \end{align*}
iff
    \begin{align*}
        E[g(Y_n)] \to E[g(Y_\infty)] \forall g\in b.c. \text{(bounded converges)}
    \end{align*}
iff 
    \begin{align*}
        & \exists \text{ a probability space supporting sequence $Y_n$' s.t. } Y'_n \overset{D}{=} Y_n \\
        & \text{and } d(Y'_n, Y_\infty) \to 0 \, a.s. \text{ as $n \to \infty$}
    \end{align*}

\subsection{Skorokhod's Representation}
Let $\mu_n$ be a sequence of probability measures on a metric space $S$ such that $\mu_n$ converges weakly to some probability measure $\mu_\infty$ on $S$ as $n \to \infty$. Suppose that the support of $\mu_\infty$ is separable. Then there exists S-valued random variables $X_n$ defined a a common probability space such that the law of $X_n$ is $\mu_n$ for all $n$ and $X_n \to X_\infty$ almost surely. 

\subsection{Brownian Motion}
A standard Wiener process(Brownian Motion) is a stochastic process $\{W_t\}$ indexed by nonnegative real numbers $t$ with the following propertyies: 
    \begin{itemize}
        \item $W_0 = 0$
        \item With probability 1, the function $t \to W_t$ is continuous in $t$ 
        \item Stationary independent increment 
        \item $W_{t+s} - W_{s}$ has $N(0, t)$ distribution 
    \end{itemize}
Some result include that if $X_n \Rightarrow X_\infty$, then it must be that $X_\infty$ is Gaussian, $E[X_\infty] = 0$ with finite variance $\sigma^2 t$. \\

Another result is that 



\subsection{Donsker's Theorem} 
Donsker's theorem is a function extension of the CLT. Let $X_i$ be iid r.v. with mean 0 and variance 1. Define the diffusively rescaled random walk as 
    \begin{align*}
        & W^{(n)}(t) := \frac{S_{\lfloor nt \rfloor}}{\sqrt{n}} \quad t\in[0,1] \\
        & S_n := \sum_{i=1}^n X_i 
    \end{align*}
    
CLT: $W^{(n)}(1)$ converges in distribution to standard Gaussian r.v. $W(1)$ as $n\to\infty$. Donsker's invariance principle extends this converges to the whole function.  \\

First notice that $S_n$ and $W^{(n)}(t)$ both have independent increment property (i.e.: $S_{m+n} - S_m $ is independent from $S_m - S_0$).  $W^{(n)}(t) \in D[0, \infty)$ due to the floor function. \\

Donsker's provides convergence in the Skorokhod space. 
    \begin{align*}
        & W^{(n)}(t) \Rightarrow W_\infty \text{ in } (D[(0, \infty), d) \\
        & W^{(n)}(t) \Rightarrow \sigma B \tag{Standard Brownian Motion}
    \end{align*}
The Skorokhod representation further says that there exists 
    \begin{align*}
        & W^{'(n)} \overset{a.s.}{\to} \sigma B'\\
        & d(W^{'(n)}, \sigma B') \overset{a.s.}{\to} 0
    \end{align*}
For $y\in D[0, \infty)$, let $h(y) = \max_{0 \leq t \leq 1}y(y)$, then we have 
    \begin{align*}
        y(W^{(n)}(t)) \Rightarrow y(\sigma B)
    \end{align*}

\subsection{Application: Quantile Estimation}
Estimate $q$ such that $F(q) = p$ with empirical estimator $F_n(Q_n) = p$. Note that $\sqrt{n}(F_n(x) - F(x)) = X_n \Rightarrow Z$ if $x$ is in finite dimension, in infinite dimension, we know that $ X_n \Rightarrow Z$ in D space (because $Q_n$) is discontinuous. 
    \begin{align*}
        \sqrt{n}\left((F_n(Q_n) - F(Q_n)) - (F_n(q) - F(q)) + (F(Q_n) - F(q)) + (F_n(q) - F(q)) \right) = \sqrt{n}(F_n(Q_n) - F(q))
    \end{align*}
Observation: 
    \begin{align*}
        & F_n(Q_n) - F(Q_n) = X_n(Q_n) \\
        & (F_n(q) - F(q)) = X_n(q) \\
        & (F(Q_n) - F(q)) = (Q_n - q)f(q) + O(f''(q)(Q_n - q)^2) \tag{Taylor Expansion} \\
        & F_n(q) - F(q) \text{can use CLT} \\
        & F_n(Q_n) \text{ is } O(\frac{1}{n})p 
    \end{align*}
So we have 
    \begin{align*}
        \sqrt{n}(X_n(Q_n) - X_n(q)) + \sqrt{n}(Q_n - q)f(q) + \sqrt{F(q)(1 - F(q))}N(0,1) = 0 
    \end{align*}
So now if we can show $X_n(Q_n) - X_n(q) \overset{P}{\to} 0$, then we can apply Slutsky Theorem and have 
    \begin{align*}
        Q_n - q \Rightarrow \frac{\sqrt{p(1-p)}}{f(q)} N(0,1)
    \end{align*}
We can show $X_n(Q_n) - X_n(q) \overset{P}{\to} 0$ by the following argument: Assume $F$ is continuous everywhere, from Densker we know that $X_n \Rightarrow Z$ in D space, and subsequently have uniform convergence in a compact set. From Skorhod Representation we know that there exists 
    \begin{align*}
        d(X_n', Z') \to 0 \text{ a.s. }
    \end{align*}
So we have 
    \begin{align*}
        & (X_n, Q_n) \Rightarrow (Z, q) \tag{From Slutksy in multiple dimension} \\
        & (X_n', Q_n') \to (Z', q) \text{ a.s.} \\
        & X_n'(Q_n') \to Z'(q) \\
        & X_n'(q) \to Z'(q) \\
        & \Rightarrow X_n'(Q_n') - X_n'(q) \to 0 a.s. \\
        & \Rightarrow X_n(Q_n) - X_n(q) \overset{P}{\to} 0
    \end{align*}
    

\subsection{Kolmogorov-Smirnov Test} 
K.S. test whether $X_1, ..., X_n$ is iid from $F$.

\paragraph{Intuition} How big is $\sup_{x\in \mathbbm{R}} \sqrt{n}\abs{F_n(x) - F(x)} = \sup_{x\in \mathbbm{R}}\abs{X_n(x)} \Rightarrow \sup_{x\in\mathbbm{R}}\abs{Z(x)}$. Then we can set up confidence interval through $P(\sup_{x} \abs{Z(x)} > w) = 0.01$. All we need to know is the limit distribution $Z$ (Densker only gives us its existence). 

\paragraph{Equivalent} Turns out that if $Z$ is the limit associate with $F$, let $\tilde{Z}$ be the limit associate with uniform $[0,1]$, then 
    \begin{align*}
        Z(x) = \tilde{Z}(F(x)) \quad \forall x
    \end{align*}
So we have 
    \begin{align*}
        \sup_x \abs{Z(x)} \overset{D}{=} \sup_x \abs{\tilde{Z}(F(x))} = \sup_{y \in [0,1]}\abs{\tilde{Z}(y)}
    \end{align*}
To summarize, as long as we can find the limit distribution $\tilde{Z}$ for uniform iid, we can apply KS test for every distribution. Turns out that such $\tilde{Z}$ has a closed form and can be easily computed. 


\subsection{Estimating Density Through Kernel Density Estimation}
\subsubsection{Formulation}
Let $X_i$ be iid from unknown distribution. We can estimate the density by put a continuous distribution (e.g.: $N(x_i, h_n^2)$) around each point $x_i$. 
    \begin{align*}
        P(N(x_i, h^2) \leq y)
        & = P(x_i + h * N(0,1) \leq y) \\
        & = P(N(0,1) \leq \frac{y - x_i}{h}) \\
        & = \Phi(\frac{y - x_i}{h}) \\
        \partiald{y}\Phi(\frac{y - x_i}{h}) 
        & = \frac{1}{h}\phi(\frac{y-x_i}{h})
    \end{align*}
The final density is the weighted sum of the density: 
    \begin{align*}
        f_n(y) = \frac{1}{h} \frac{1}{n} \sum_{i=1}^n \phi(\frac{y-x_i}{h}) 
    \end{align*}

\subsubsection{Convergence}
In general, we can decompose MSE: 
    \begin{align*}
        E[f_n(y) - f(y)]^2 
        & = E\left[f_n(y) - E[f_n(y)] + E[f_n(y)] - f(y)\right]^2 \\
        & = E[f_n(y) - E[f_n(y)]]^2 + E[E[f_n(y)] - f(y)]^2 + 2 * E[f_n(y) - E[f_n(y)]] E[E[f_n(y)] - f(y)] \\
        & = E[f_n(y) - E[f_n(y)]]^2 + E[E[f_n(y)] - f(y)]^2 \\
        & = Var(f_n(y)) + Bias^2
    \end{align*}
The expected value of the estimator is : 
    \begin{align*}
        E[f_n(y)] 
        & = E[\phi(\frac{y - x_1}{h}) \frac{1}{h}]\\
        & = \int_{-\infty}^\infty \phi(\frac{y-x}{h}) \frac{1}{h} f(x) \, dx \\
        & = \int_{-\infty}^\infty \phi(z) f(y-zh) \, dz \tag{$z = \frac{y-x}{h}$} \\
        & = \int_{-\infty}^\infty \phi(z)\left( f(y) - zh f'(y) + \frac{z^2 h^2}{2} f''(y) + o(h^2) \right) \, dz \tag{Taylor Expansion} \\
        & = f(y) \int_{-\infty}^\infty \phi(z) \, dz - f'(y)h\int_{-\infty}^\infty z \phi(z) \, dz + \frac{f''(y)h^2}{2}\int_{-\infty}^\infty z^2 \phi(z) \, dz + o(h^2) \\
        & = f(y) - f'(y)h\int_{-\infty}^\infty z \phi(z) \, dz + \frac{f''(y)h^2}{2}\int_{-\infty}^\infty z^2 \phi(z) \, dz + o(h^2)  \tag{Integral of a pdf}\\
        & = f(y) + \frac{f''(y)h^2}{2}\int_{-\infty}^\infty z^2 \phi(z) \, dz + o(h^2) \tag{Property of $\phi$} \\
        & = f(y) + \frac{h^2}{2}f''(y) + o(h^2) \tag{Property of $\phi$} \\
    \end{align*}
So the intuition is that we want to have small $h$, as little smoothing as possible. The variance of the estimator is: 
    \begin{align*}
        \frac{1}{n} Var\left(\frac{1}{h} \phi(\frac{y-x_1}{h}) \right) 
        & = \frac{1}{nh^2} Var\left(\phi(\frac{y-x_1}{h}) \right)\\
        & = \frac{1}{nh^2} \left(E[\phi^2(\frac{y-x_1}{h})] - E[\phi(\frac{y-x_1}{h})]^2 \right) \\
        & = \frac{1}{nh^2} \left(\int_{-\infty}^\infty \phi^2(\frac{y-x}{h}) f(x) \, dx - E[\phi(\frac{y-x_1}{h})]^2 \right)\\
        & = \frac{1}{n} \left( \frac{1}{h}\int_{-\infty}^\infty \phi^2(z) f(y-zh) \, dz - \frac{1}{h}E[\phi(\frac{y-x_1}{h})]^2 \right)\\
        & = \frac{1}{n}(\frac{1}{h} O(1) + O(h)) \\
        & \approx \frac{1}{nh} \text{ as } h \to 0
    \end{align*}
So the variance is $O(\frac{1}{nh})$, which means we want to have $h$ big, as much smoothing as possible. 

\subsubsection{Bias Variance Tradeoff} 
Given above result, we can set the bias squared equal to variance to get the optimal $h$ 
    \begin{align*}
    \frac{1}{nh} \approx h^4 \Longrightarrow h \approx n^{-1/5}
    \end{align*}
The optimal RMSE under this condition is 
    \begin{align*}
        \frac{1}{nh} + h^4 = \sqrt{n^{-4/5}} = n^{-2/5}
    \end{align*}
The optimal $h$ will give us CLT $n^{2/5}(f_n(y)-f(y))\Rightarrow N(a, b^2)$. In reality, bias is hard to estimate and the variance is easier to estimate. So we might want to set $h > n^{-1/5}$ to have a low bias high variance estimator. 


\subsection{Estimating Mode} 
Mode is defined as 
    \begin{align*}
        m = \argmax_x pdf(x)
    \end{align*}
The intuition is to estimate $f'(x)$ through: 
    \begin{align*}
        & \frac{f_n(y+r_n) - f_n(y)}{r_n}  \\
        & h_n \to 0 \tag{Used for estimate $f_n$} \\
        & r_n \to 0
    \end{align*}
But notice tha t as $r_n\to0$, the variance of this estimator gets big. So we need to let $h_n \to 0$ slower compare to the previous optimal $h$. In general, as we estimate high order derivative of pdf, we depend more on local structure and requires smoother curve, i.e.: bigger $h$. 


\subsection{CDF Estimation: Integrating PDF vs. Plug In} 
So far we have two CDF estimator: 
    \begin{align*}
        & F_n(y) = \frac{1}{n}\sum \indic{x_i \leq y} \tag{Plug In Estimator $h=0$} \\
        & \hat{F}_n(y) = \int_{-\infty}^y f_n(z) \, dz \tag{pdf Estimator $h \propto n^{-1/5}$}
    \end{align*}
Here we demonstrate why the plug in estimator is better: 
    \begin{align*}
        \hat{F}_n(y)
        & = \frac{1}{n} \sum_{i=1}^n \int_{-\infty}^y \phi(\frac{z-x_i}{h}) \frac{1}{h} \, dz \\
        & = \frac{1}{n} \sum_{i=1}^n \Phi(\frac{y-x_i}{h})
    \end{align*}
Notice that as $h \to 0$, $\Phi(\cdot)$ will either goes to 1 (if $x_i < y$) or to 0 otherwise. So the function is bounded and we can apply bounded convergence theorem: 
    \begin{align*}
        & E[\Phi(\frac{y-x_i}{h})] \to E[\indic{x_i \leq y}] = P(X_i \leq y) \\
        & E[\Phi^2(\frac{y-x_i}{h})] \to P(X_i \leq y) \tag{By the same 0, 1 argument} 
    \end{align*}
We can then look at variance of the estimator: 
    \begin{align*}
        Var(F_n(y))
        & = \frac{1}{n} Var(\indic{X_i \leq y}) \\
        Var(\hat{F}_n(y)) 
        & = \frac{1}{n} Var(\Phi(\frac{y-x_i}{h})) \\
        & \to \frac{1}{n}(P(X_i \leq y) - P^2(X_i \leq y))
    \end{align*}
Now for bias: 
    \begin{align*}
        E[\hat{F}_n(y) - F(y)]
        & = E[\Phi(\frac{y-x}{h})] - F(y) \\
        & = \int_{-\infty}^\infty \Phi(\Phi(\frac{y-x}{h}) f(x) \, dx - F(y) \\
        & = h\int_{-\infty}^\infty \Phi(z) f(y-zh) \, dz - F(y) \\
        & = h\left(\int_{-\infty}^0 \Phi(z)f(y-zh) \, dz + \int_{0}^\infty \Phi(z)f(y-zh) \, dz \right) - F(y) \\
        & = h\left(\int_{-\infty}^0 \Phi(z)f(y-zh) \, dz + \int_{0}^\infty (1 - \bar{\Phi}(z))f(y-zh) \, dz \right) - F(y) \tag{$\bar{\Phi}(x) \equiv 1 - \Phi(x)$} \\
        & = h\left(\int_{-\infty}^0 \Phi(z)f(y-zh) \, dz - \int_{y}^{-\infty} (1 - \bar{\Phi}(\frac{y-\omega}{h}))f(\omega) \frac{1}{h} \, d\omega \right) - F(y) \\
        & =  h\left(\int_{-\infty}^0 \Phi(z)f(y-zh) \, dz + \int_{y}^{-\infty}  \bar{\Phi}(\frac{y-\omega}{h})f(\omega) \frac{1}{h} \, d\omega \right)  \\
        & =  h\left(\int_{-\infty}^0 \Phi(z)f(y-zh) \, dz + \int_{0}^{\infty}  \bar{\Phi}(z)f(y-zh) \, dz \right)  \\
        & =  h\left(\int_{-\infty}^0 \Phi(z)f(y-zh) \, dz + \int_{0}^{-\infty}  \bar{\Phi}(-t)f(y+th) * -1 \, dt \right)  \\
        & =  h\left(\int_{-\infty}^0 \Phi(z)f(y-zh) \, dz + \int_{0}^{-\infty}  \Phi(t)f(y+th) \, dt \right) \tag{$\bar{\Phi(-x)}=\Phi(x)$} \\
        & =  h\left(\int_{-\infty}^0 \Phi(z)(f(y-zh) - f(y + zh)) \, dz \right)
    \end{align*}
So bias squared is in $h^2 = n^{-2/5}$ term, and variance is in $\frac{1}{n}$ term.

\subsection{Multidimensional density estimation with Multivariate Normal Smoother} 
Result: bias $O(h^2)$ and variance $O(\frac{c}{nh^d})$ where $d$ is the dimension. RMSE is $O(n^{-2/(d+4)})$. So as $d$ increases, we will have less and less convergence. 


\section{Bayesian Statistics} 
\subsection{General Formulation} 
    \begin{align*}
        f(p|data) = \frac{p(data|p) * f(p) }{p(data)}
    \end{align*}
On the top is the data's likelihood times the prior. The bottom is the likelihood of the data times prior integrated over entire support. 

\subsection{Example}
Setup: Bernoulli trial with result 1, 1, 1, 1. Let $f(p) = unif[0,1]$ to be the prior (uninformative prior). The posterior distribution is :
    \begin{align*}
        f(p|data) 
        & = \frac{p(data|p) * p(p)}{p(data)} \\
        & = \frac{p^4 * 1}{\int_0^1 1 * p^4 \, dp} \\
        & = 5p^4
    \end{align*}
So the posterior mode is 
    \begin{align*}
        \argmax_p 5p^4 = 1
    \end{align*}
The posterior mean is 
    \begin{align*}
    \int_0^1 p * f(p|data) \,dp = \int_0^1 5p^5\, dp = \frac{5}{6}
    \end{align*}

\subsection{Beta Distribution as Conjugate Prior for Binomial Distribution} 
Beta distribution with parameter $\alpha, \beta$ is defined as:
    \begin{align*}
        f(p) = \frac{p^{\alpha-1}(1-p)^{\beta-1}}{Beta(\alpha, \beta)}
    \end{align*}
If an experiment has $k$ success and $n-k$ failure. The likelihood of the data is: 
    \begin{align*}
        p^k (1-p)^{n-k}
    \end{align*}
The posterior can always be looked at from proportion perspective (since $p(data)$ is a constant): 
    \begin{align*}
    f(p|data) 
    & \propto p(data|p) * f(p) \\
    & = p^k (1-p)^{n-k} p^{\alpha-1}(1-p)^{\beta-1}\\
    & = p^{\alpha + k - 1} (1-p)^{\beta + n - k - 1}
    \end{align*}
So the posterior distribution is also a new beta density with $\alpha' = \alpha + k$ and $\beta' = \beta + n - k$

\subsection{Gaussian Distribution as Conjugate Prior for Gaussian} 
Let $X_i$ be iid from $N(\mu^*,1)$, and assume the prior distribution for $\mu^*$ to be $N(0,1)$. The posterior is 
    \begin{align*}
        f(\mu|data) 
        & \propto f(data|\mu) *f(\mu) \\
        & = (\frac{1}{\sqrt{2\pi}})^n\exp(-\sum \frac{(x_i-\mu)^2}{2}) \frac{1}{\sqrt{2\pi}}\exp(-\mu^2/2)\\
        & \propto \exp(-\sum \frac{(x_i-\mu)^2}{2}) \exp(-\mu^2/2) \\
        & =\exp(-\frac{1}{2} (\mu^2 + \sum x_i^2 + n*\mu^2 - 2\mu \sum x_i))\\
        & \propto \exp(-\frac{1}{2} ((n+1)\mu^2 - 2\mu \sum x_i))\\
        & = \exp(-\frac{n+1}{2} (\mu^2 - \frac{2\mu \sum x_i)}{n+1})\\
        & \propto \exp(-\frac{n+1}{2}(\mu-\frac{\sum x_i}{n+1})^2)
    \end{align*}
So this is a new Gaussian distribution with $\mu' = \frac{\sum x_i}{n+1}$ and $\sigma' = \frac{1}{n+1}$

\subsection{Gamma Distribution as Conjugate Prior} 
Overall result: If $X_i$ are iid from $Gamma(\lambda, \alpha)$, then there exists a conjugate prior on $\lambda^*$ which is Gamma distribution.  


\subsection{Bayesian Regression}

\subsubsection{Non-Bayesian Regression Formulation}
Regression normally has the form of 
    \begin{align*}
        Y_i = \alpha_0 x_0 + \alpha_1 x_1 + ... + \alpha_d x_d + \epsilon_i
    \end{align*}
where $x_0 = 1$ and $\epsilon_i$ is iid $N(0, \sigma^2)$. The goal is to estimate $\alpha_i$ and $\sigma$. By setting the equation to be the form of $\epsilon_i = \cdot$, we can get Gaussian likelihood 
    \begin{align*}
        L(\alpha_0, ..., \alpha_d, \sigma^2) 
        & = \prod_{i=1}^n\left(\frac{1}{\sqrt{2\pi}\sigma} \exp\left(-\frac{1}{2}(\frac{y_i - \sum_{j=0}^{d} \alpha_j x_{ji}}{\sigma})^2  \right) \right) \\
        l(\alpha_0, ..., \alpha_d, \sigma^2)
        & = \log(\frac{1}{\sqrt{2\pi}\sigma}) - \sum_{i=0}^n \frac{1}{2} \left(\frac{y_i - \sum_{j=0}^d \alpha_j x_{ji}}{\sigma} \right)^2 \\
    \end{align*}
Notice that the $\alpha_{j}$ that maximizes the log likelihood is equivalent to minimize $\sum_{i=0}^n(y_i - \sum_{j=0}^{d} \alpha_j x_{ji})^2$.\\
In matrix vector formulation, we let $Y$ and $a$ be a $n \times 1$ matrix, $X$ is $n\times d$ matrix. So we have 
    \begin{align*}
        & Y = Xa + \Sigma \tag{$\Sigma \sim N(0, \sigma^2 C)$}\\
        & \text{$C$ in general is a known typically diagonal non-identity matrix}
    \end{align*}
Then the MLE estimator is equivalent to minimize 
    \begin{align*}
        (Y-Xa)^T C^{-1} (Y-Xa)
    \end{align*}
    
\subsubsection{Bayesian Regression Formulation}
In Bayesian formulation, we have 
    \begin{align*}
        & Y = Xa + \Sigma\\
        & \Sigma \sim N(0, \sigma^2 C) \tag{$C$ is known, but $\sigma^2$ is unknown} \\
        & a \sim N(O, I) \tag{Gaussian Prior} 
    \end{align*}
From this formulation, we know that the posterior is 
    \begin{align*}
        f(\alpha|data) \propto \exp\left(-\frac{\sum_{j=0}^d \alpha_i^2}{2} - \frac{(Y-X\alpha)^TC^{-1}(Y-X\alpha)}{2\sigma^2} \right)
    \end{align*}
Notice that the first part is from $f(a)$, and the second part is from $p(data|a)$. So we can derive the mode of the posterior (the $a$ that maximizes the probability) to be 
    \begin{align*}
        \argmin_\alpha (Y-X\alpha)^T C^{-1} (Y-X\alpha) + \frac{1}{2}\sum_{j=0}^d \alpha_i^2
    \end{align*}
Hence Regression with Gaussian prior is equivalent to regression with L2 regularization. \\

If we use Laplace distribution as prior, such as $\frac{1}{2}e^{-\abs{\alpha}}$, then the final mode is equivalent to L1 regularization. 
    \begin{align*}
        \argmin_\alpha (Y-X\alpha)^T C^{-1} (Y-X\alpha) + \frac{1}{2}\sum_{j=0}^d \abs{\alpha_i}
    \end{align*}        
    
